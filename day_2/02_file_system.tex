%=============================================================================%
% Author: 	Bram Kuijper
% Date: 	20/03/2019
% Title: 	Python workshop: strings and regex
%=============================================================================%

%=============================================================================%
% Preamble
%=============================================================================%
% Libraries

\documentclass[xcolor=table]{beamer}
\usepackage{beamerthemeshadow}
\usepackage[T1]{fontenc}
\usepackage{helvet}
\usepackage[]{graphicx}
\usepackage{array}
\usepackage{color}
\definecolor{dkgreen}{rgb}{0,0.6,0}
\definecolor{gray}{rgb}{0.5,0.5,0.5}
\definecolor{mauve}{rgb}{0.58,0,0.82}
\definecolor{deepblue}{rgb}{0,0,0.5}
\definecolor{deepred}{rgb}{0.6,0,0}
\definecolor{deepgreen}{rgb}{0,0.5,0}
\definecolor{lightgray}{rgb}{0.92,0.92,0.92}
\usepackage{listings} % to insert code
\usepackage{textpos} % textblock
\usepackage{textcomp} % textblock
\usepackage{hyperref}
\hypersetup{colorlinks=true, urlcolor=blue, linkcolor=black} 
% Listing set up
% bash
\lstdefinestyle{bash}{
language=bash,                     % the language of the code
basicstyle=\scriptsize\ttfamily,       % the size of the fonts that are used for the code
numbers=none,%left,                   % where to put the line-numbers
numberstyle=\tiny\color{gray},  % the style that is used for the line-numbers
stepnumber=1,                   % the step between two line-numbers. If it's 1, each line
                          % will be numbered
numbersep=5pt,                  % how far the line-numbers are from the code
backgroundcolor=\color{lightgray},  % choose the background color. You must add \usepackage{color}
showspaces=false,               % show spaces adding particular underscores
showstringspaces=false,         % underline spaces within strings
showtabs=false,                 % show tabs within strings adding particular underscores
frame=lines,%single,                   % adds a frame around the code
rulecolor=\color{black},        % if not set, the frame-color may be changed on line-breaks within not-black text (e.g. commens (green here))
tabsize=2,                      % sets default tabsize to 2 spaces
captionpos=b,                   % sets the caption-position to bottom
breaklines=true,                % sets automatic line breaking
breakatwhitespace=false,        % sets if automatic breaks should only happen at whitespace
title=\lstname,                 % show the filename of files included with \lstinputlisting;
                          % also try caption instead of title
keywordstyle=\color{blue},      % keyword style
commentstyle=\color{dkgreen},   % comment style
stringstyle=\color{mauve},      % string literal style
escapeinside={\%*}{*)},         % if you want to add a comment within your code
morekeywords={}            % if you want to add more keywords to the set
}

% Python
\lstdefinestyle{python}{
language=python,
formfeed=\newpage,
basicstyle=\scriptsize\ttfamily,
commentstyle=\color{deepgreen},%\color{gray},
numbers=left,
numberstyle=\tiny\color{gray},
stepnumber=1,
numbersep=5pt,
extendedchars=true,
inputencoding=utf8x,
backgroundcolor=\color{lightgray},%\color{white},
showspaces=false,
showstringspaces=false,
showtabs=false,
frame=lines,
upquote=true,
tabsize=4,
captionpos=b,
breaklines=true,
breakatwhitespace=false,
title=\lstname,
escapeinside=||,
keywordstyle=\color{deepblue},
emphstyle=\color{deepred},
stringstyle=\color{mauve},
literate={ö}{{\"o}}1
       {ä}{{\"a}}1
       {ü}{{\"u}}1
       {ç}{{\c{c}}}1
%morekeywords={models, lambda, forms}
}

\graphicspath{ {../img/} }
\title[Python for scientific research]{Python for scientific research}
\subtitle{Input, output and the filesystem}
\author{Bram Kuijper}
\institute[]{University of Exeter, Penryn Campus, UK}
\titlegraphic{
\hfill
\includegraphics[width=\textwidth, keepaspectratio]{logo.jpg}}
%=============================================================================%
%=============================================================================%
% Start of Document
%=============================================================================%
%=============================================================================%
\begin{document}

%=============================================================================%
%=============================================================================%
\begin{frame}
\titlepage
\end{frame}

%=============================================================================%
%=============================================================================%
\begin{frame}{What we've done so far}

	\begin{enumerate}\addtolength{\itemsep}{1\baselineskip}
		\item Declare variables using built-in data types and execute operations
		on them
		\item Use flow control commands to dictate the order in which commands are run
		and when
		\item Encapsulate programs into reusable functions, modules and packages
        \item Work with textual data and pattern matching
        \item \textbf{Next} interact with the file system
	\end{enumerate}

\end{frame}

\begin{frame}[fragile]
    \frametitle{Writing files}
\begin{itemize}
    \item open a file for writing, using the \href{https://docs.python.org/3/library/functions.html#open}{\texttt{open()}} command \pause
\begin{lstlisting}[style=python]
# the 'w' flag reflects that we write to a file
# any existing contents will be overwritten
the_file_object = open("my_first_file","w") | \pause |

# write a string to the file
the_file_object.write("Hello world!") | \pause |

# always close the file
the_file_object.close()
\end{lstlisting}
\end{itemize}
\end{frame}

%=============================================================================%
%=============================================================================%

\begin{frame}[fragile]
    \frametitle{Reading files}
\begin{itemize}
    \item open a file for reading, using the \href{https://docs.python.org/3/library/functions.html#read}{\texttt{read()}} command \pause
\begin{lstlisting}[style=python]
# the 'r' flag reflects that we read from a file
the_file_object = open("my_first_file","r") | \pause |

# get the file contents as a string
file_contents = the_file_object.read() | \pause |

# always close the file
the_file_object.close() | \pause |

# process file output
print(file_contents) # Hello World!
\end{lstlisting}
\end{itemize}
\end{frame}

%=============================================================================%
%=============================================================================%

\begin{frame}[fragile]
    \frametitle{Other file operations}
\begin{itemize}
    \item File operations can be specified by the flag provided to \href{https://docs.python.org/3/library/functions.html#open}{\texttt{open()}} function \pause 
\begin{center}
    \rowcolors{1}{}{lightgray} %-- this indicates the change in odd and pair rows
    \begin{tabular}{|l|l|}\hline
        Flag & File operation \\ \hline
        \texttt{"w"}
    & Write to a file, file will be truncated first \\
        \texttt{"r"} & Reading from a file \\
        \texttt{"r+"} & Reading and writing to a file \\
        \texttt{"a"} & Append to a file \\
        \texttt{"a+"} & Read from and write (by appending) to a file \\
        \texttt{"x"} & Exclusive creation, fails if file exists \\ \hline
    \end{tabular}
\end{center}
\end{itemize}
\end{frame}
% End of Document
%=============================================================================%
%=============================================================================%
\end{document}
