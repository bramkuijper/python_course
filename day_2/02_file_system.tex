%=============================================================================%
% Author: 	Bram Kuijper
% Date: 	20/03/2019
% Title: 	Python workshop: strings and regex
%=============================================================================%

%=============================================================================%
% Preamble
%=============================================================================%
% Libraries

\documentclass[xcolor=table]{beamer}
\usepackage{beamerthemeshadow}
\usepackage[T1]{fontenc}
\usepackage{helvet}
\usepackage[]{graphicx}
\usepackage{array}
\usepackage{color}
\definecolor{dkgreen}{rgb}{0,0.6,0}
\definecolor{gray}{rgb}{0.5,0.5,0.5}
\definecolor{mauve}{rgb}{0.58,0,0.82}
\definecolor{deepblue}{rgb}{0,0,0.5}
\definecolor{deepred}{rgb}{0.6,0,0}
\definecolor{deepgreen}{rgb}{0,0.5,0}
\definecolor{lightgray}{rgb}{0.92,0.92,0.92}
\usepackage{listings} % to insert code
\usepackage{textpos} % textblock
\usepackage{textcomp} % textblock
\usepackage{hyperref}
\usepackage{dirtree}
\hypersetup{colorlinks=true, urlcolor=blue, linkcolor=black} 
% Listing set up
% bash
\lstdefinestyle{bash}{
language=bash,                     % the language of the code
basicstyle=\scriptsize\ttfamily,       % the size of the fonts that are used for the code
numbers=none,%left,                   % where to put the line-numbers
numberstyle=\tiny\color{gray},  % the style that is used for the line-numbers
stepnumber=1,                   % the step between two line-numbers. If it's 1, each line
                          % will be numbered
numbersep=5pt,                  % how far the line-numbers are from the code
backgroundcolor=\color{lightgray},  % choose the background color. You must add \usepackage{color}
showspaces=false,               % show spaces adding particular underscores
showstringspaces=false,         % underline spaces within strings
showtabs=false,                 % show tabs within strings adding particular underscores
frame=lines,%single,                   % adds a frame around the code
rulecolor=\color{black},        % if not set, the frame-color may be changed on line-breaks within not-black text (e.g. commens (green here))
tabsize=2,                      % sets default tabsize to 2 spaces
captionpos=b,                   % sets the caption-position to bottom
breaklines=true,                % sets automatic line breaking
breakatwhitespace=false,        % sets if automatic breaks should only happen at whitespace
title=\lstname,                 % show the filename of files included with \lstinputlisting;
                          % also try caption instead of title
keywordstyle=\color{blue},      % keyword style
commentstyle=\color{dkgreen},   % comment style
stringstyle=\color{mauve},      % string literal style
escapeinside={\%*}{*)},         % if you want to add a comment within your code
morekeywords={}            % if you want to add more keywords to the set
}

% Python
\lstdefinestyle{python}{
language=python,
formfeed=\newpage,
basicstyle=\scriptsize\ttfamily,
commentstyle=\color{deepgreen},%\color{gray},
numbers=left,
numberstyle=\tiny\color{gray},
stepnumber=1,
numbersep=5pt,
extendedchars=true,
inputencoding=utf8x,
backgroundcolor=\color{lightgray},%\color{white},
showspaces=false,
showstringspaces=false,
showtabs=false,
frame=lines,
upquote=true,
tabsize=4,
captionpos=b,
breaklines=true,
breakatwhitespace=false,
title=\lstname,
escapeinside=||,
keywordstyle=\color{deepblue},
emphstyle=\color{deepred},
stringstyle=\color{mauve},
literate={ö}{{\"o}}1
       {ä}{{\"a}}1
       {ü}{{\"u}}1
       {ç}{{\c{c}}}1
%morekeywords={models, lambda, forms}
}

\graphicspath{ {../img/} }
\title[Python for scientific research]{Python for scientific research}
\subtitle{Input, output and the filesystem}
\author{Bram Kuijper}
\institute[]{University of Exeter, Penryn Campus, UK}
\titlegraphic{
\hfill
\includegraphics[width=\textwidth, keepaspectratio]{logo.jpg}}
%=============================================================================%
%=============================================================================%
% Start of Document
%=============================================================================%
%=============================================================================%
\begin{document}

%=============================================================================%
%=============================================================================%
\begin{frame}
\titlepage
\end{frame}

%=============================================================================%
%=============================================================================%
\begin{frame}{What we've done so far}
	\begin{enumerate}\addtolength{\itemsep}{1\baselineskip}
		\item Declare variables using built-in data types and execute operations
		on them
		\item Use flow control commands to dictate the order in which commands are run
		and when
		\item Encapsulate programs into reusable functions, modules and packages
        \item Work with textual data and pattern matching
        \item \textbf{Next} working with files \& the file system 
	\end{enumerate}

\end{frame}

%=============================================================================%
%=============================================================================%

\begin{frame}[fragile]
    \frametitle{Working with the filesystem: key modules}
\begin{itemize}\addtolength{\itemsep}{\baselineskip}
    \item Work with \emph{names} of directories and files (pathnames) \pause
    \begin{itemize}
        \item Newer, object-oriented \href{https://docs.python.org/3/library/pathlib.html#module-pathlib}{\texttt{Pathlib}} module (Python 3.5+) \pause
        \item This course: conventional \href{https://docs.python.org/3.7/library/os.path.html}{\texttt{os.path}} module \pause
    \end{itemize}
\item Access the filesystem (disks, file descriptors, directories, etc): \pause
    \begin{itemize}
        \item Basic filesystem tasks: the \href{https://docs.python.org/3.7/library/os.html#module-os}{\texttt{os}} module (user access rights, removing directories, etc) \pause
        \item High-level operations: the \href{https://docs.python.org/3/library/shutil.html}{\texttt{shutil}} module (copying files, removing directory trees, etc)  \pause
    \end{itemize}
\item Reading and writing files
    \begin{itemize}
        \item The \href{https://docs.python.org/3/library/functions.html#open}{\texttt{open()}} command \pause
    \end{itemize}
\end{itemize}
\end{frame}

%=============================================================================%
%=============================================================================%

\begin{frame}[fragile]
    \frametitle{Working with filesystems: key functions}
\begin{itemize}\addtolength{\itemsep}{\baselineskip}
        \item Some important functions of the \href{https://docs.python.org/3.7/library/os.html#module-os}{\texttt{os}} module:\pause
            \begin{itemize}\addtolength{\itemsep}{\baselineskip}
                \item \href{https://docs.python.org/3.7/library/os.html#os.getcwd}{\texttt{os.getcwd()}} obtains current working directory \pause
                \item \href{https://docs.python.org/3.7/library/os.html#os.chdir}{\texttt{os.chdir(dir)}} changes current working directory to \texttt{dir} \pause
                \item \href{https://docs.python.org/3.7/library/os.html#os.rmdir}{\texttt{os.rmdir(dir)}} removes directory \texttt{dir} \pause
                \item \href{https://docs.python.org/3.7/library/os.html#os.mkdir}{\texttt{os.mkdir(dir)}} makes directory \texttt{dir} \pause
                \item \href{https://docs.python.org/3.7/library/os.html#os.listdir}{\texttt{os.listdir(dir=".")}} list files in directory \texttt{dir} \pause
                \item \href{https://docs.python.org/3.7/library/os.html#os.walk}{\texttt{os.walk(dir)}} loop through all subdirectories and files in directory \texttt{dir} \pause
        \end{itemize}
    \end{itemize}
\end{frame}

\begin{frame}[fragile]
    \frametitle{Working with filesystems: key functions II}
        \begin{itemize}\addtolength{\itemsep}{\baselineskip}
            \item Some important functions of the \href{https://docs.python.org/3.7/library/os.html#module-os.path}{\texttt{os.path}} module:\pause
                \begin{itemize}\addtolength{\itemsep}{\baselineskip}
                        \item \href{https://docs.python.org/3.7/library/os.path.html#os.path.exists}{\texttt{os.path.exists(path)}} checks whether path exists \pause
                        \item \href{https://docs.python.org/3.7/library/os.path.html#os.path.expanduser}{\texttt{os.path.expanduser("\textasciitilde")}} provides home directory name of current user \pause
                        \item \href{https://docs.python.org/3.7/library/os.path.html#os.path.join}{\texttt{os.path.join(path1,path2)}} concatenates path1 and path2  \pause
                        \item \href{https://docs.python.org/3.7/library/os.path.html#os.path.split}{\texttt{os.path.split(path)}} splits path \texttt{a/b/c.txt} into \texttt{(a/b, c.txt)} \pause
                        \item \href{https://docs.python.org/3.7/library/os.path.html#os.path.splitext}{\texttt{os.path.splitext(path)}} splits path \texttt{a/b/c.ext} into \texttt{(a/b/c, ext)} \pause
                \end{itemize}
        \end{itemize}
\end{frame}


\begin{frame}[fragile]
    \frametitle{Working with paths 1}
    \begin{itemize}
        \item Task: what is the current working directory? 
            \begin{itemize}
                \item Access the filesystem using the \href{https://docs.python.org/3.7/library/os.html#module-os}{\texttt{os}} module: \pause
            \end{itemize}
\begin{lstlisting}[style=python]
import os # import modules | \pause |

wdir = os.getcwd() # returns pathname (as string) | \pause |
print(wdir) # e.g., "C:\" 
\end{lstlisting}
    \end{itemize}
\end{frame}

\begin{frame}[fragile]
    \frametitle{Working with paths 2}
    \begin{itemize}
        \item Task: change the current working directory to the home directory: \pause
    \begin{itemize}
            \item Access the filesystem using the \href{https://docs.python.org/3.7/library/os.html#module-os}{\texttt{os}} module 
            \item Manipulate path names using the \href{https://docs.python.org/3.7/library/os.path.html}{\texttt{os.path}} module \pause
    \end{itemize}
\begin{lstlisting}[style=python]
import os, os.path # import modules | \pause |

# Get homedir through expanduser()
# "~" denotes homedir in Unix (but this works on Windows too)
home_dir = os.path.expanduser("~") | \pause |
print(home_dir) # "C:\Users\foo323" | \pause |

# change the directory to the homedir
os.chdir(home_dir) | \pause |

# check the result
print(os.getcwd()) # "C:\Users\foo323"
\end{lstlisting}
    \end{itemize}
\end{frame}

%\begin{frame}[fragile]
%    \frametitle{Finding files}
%            \dirtree{%
%    .1 C:.
%    .2 Users.
%    .2 Public.
%    .3 foo323.
%    .4 Desktop.
%    .4 Downloads.
%    .4 Saved Games.
%    .4 Photos 2018.
%    .4 Scripts.
%            }
%    \begin{itemize}
%        \item Task: in your home directory, find all directories (non-nested) whose names contain whitespace 
%    \end{itemize}
%\end{frame}

\begin{frame}[fragile]
    \frametitle{Finding files}
    \begin{itemize}
        \item Task: in your home directory, find all directories (non-nested) which contain whitespace
\begin{lstlisting}[style=python]
import re, os, os.path # import the modules | \pause |

# go to home dir
home_dir = os.path.expanduser("~") | \pause |
os.chdir(home_dir) | \pause |

# list all files in home dir
file_list = os.listdir() # ["Desktop", "Downloads", ...] | \pause |

# declare list containing dirs with spaces
dirs_with_spaces = [] | \pause |

for file_i in file_list: | \pause |
    # check for white space in filename and whether file is directory
    if re.search(r"\s",file_i) is not None and os.path.isdir(file_i): | \pause |
        dirs_with_spaces += [file_i] # append
\end{lstlisting} \pause
\item \href{https://docs.python.org/3.7/library/os.html#os.listdir}{os.listdir()} retrieves a list of files in the current directory (not of any subdirectories)
    \end{itemize}
\end{frame}


\begin{frame}[fragile]
    \frametitle{Finding files 2}
    \begin{itemize}
        \item Task: list all files \emph{anywhere} within the home directory which have a size larger than 50 kB \pause
        \item We now need to iterate over all files, using \href{https://docs.python.org/3.7/library/os.html#os.walk}{os.walk()}
    \begin{itemize}
        \item \href{https://docs.python.org/3.7/library/os.html#os.walk}{os.walk()} walks the directory tree, returning a tuple for each directory with \texttt{(parent_dir, subdirectories, files)}

\begin{lstlisting}[style=python]
import re, os, os.path # import the modules | \pause |

# go to home dir
home_dir = os.path.expanduser("~") | \pause |
os.chdir(home_dir) | \pause |

# list all files in home dir
file_list = os.listdir() # ["Desktop", "Downloads", ...] | \pause |

# declare list containing dirs with spaces
dirs_with_spaces = [] | \pause |

for file_i in file_list: | \pause |
    # check for white space in filename and whether file is directory
    if re.search(r"\s",file_i) is not None and os.path.isdir(file_i): | \pause |
        dirs_with_spaces += [file_i] # append
\end{lstlisting}
    \end{itemize}
\end{frame}

\begin{frame}[fragile]
    \frametitle{Writing files}
\begin{itemize}
    \item open a file for writing, using the \href{https://docs.python.org/3/library/functions.html#open}{\texttt{open()}} command \pause
\begin{lstlisting}[style=python]
# the 'w' flag reflects that we write to a file
# any existing contents will be overwritten
the_file_object = open("my_first_file","w") | \pause |

# write a string to the file
the_file_object.write("Hello world!") | \pause |

# always close the file
the_file_object.close()
\end{lstlisting}
\end{itemize}
\end{frame}

%=============================================================================%
%=============================================================================%

\begin{frame}[fragile]
    \frametitle{Reading files}
\begin{itemize}
    \item open a file for reading, using the \href{https://docs.python.org/3/library/functions.html#read}{\texttt{read()}} command \pause
\begin{lstlisting}[style=python]
# the 'r' flag reflects that we read from a file
the_file_object = open("my_first_file","r") | \pause |

# get the file contents as a string
file_contents = the_file_object.read() | \pause |

# always close the file
the_file_object.close() | \pause |

# process file output
print(file_contents) # Hello World!
\end{lstlisting}
\end{itemize}
\end{frame}

%=============================================================================%
%=============================================================================%

\begin{frame}[fragile]
    \frametitle{Other file operations}
\begin{itemize}
    \item File operations can be specified by the flag provided to \href{https://docs.python.org/3/library/functions.html#open}{\texttt{open()}} function \pause 
\begin{center}
    \rowcolors{1}{}{lightgray} %-- this indicates the change in odd and pair rows
    \begin{tabular}{|l|l|}\hline
        Flag & File operation \\ \hline
        \texttt{"w"}
    & Write to a file, file will be truncated first \\
        \texttt{"r"} & Reading from a file \\
        \texttt{"r+"} & Reading and writing to a file \\
        \texttt{"a"} & Append to a file \\
        \texttt{"a+"} & Read from and write (by appending) to a file \\
        \texttt{"x"} & Exclusive creation, fails if file exists \\ \hline
    \end{tabular}
\end{center}
\end{itemize}
\end{frame}
% End of Document
%=============================================================================%
%=============================================================================%
\end{document}
