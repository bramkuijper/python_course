%=============================================================================%
% Author: 	Bram Kuijper
% Date: 	20/03/2019
% Title: 	Python workshop: strings and regex
%=============================================================================%

%=============================================================================%
% Preamble
%=============================================================================%
% Libraries

\documentclass[xcolor=table]{beamer}
\usepackage{beamerthemeshadow}
\usepackage[T1]{fontenc}
\usepackage{helvet}
\usepackage[]{graphicx}
\usepackage{array}
\usepackage{color}
\definecolor{dkgreen}{rgb}{0,0.6,0}
\definecolor{gray}{rgb}{0.5,0.5,0.5}
\definecolor{mauve}{rgb}{0.58,0,0.82}
\definecolor{deepblue}{rgb}{0,0,0.5}
\definecolor{deepred}{rgb}{0.6,0,0}
\definecolor{deepgreen}{rgb}{0,0.5,0}
\definecolor{lightgray}{rgb}{0.92,0.92,0.92}
\usepackage{listings} % to insert code
\usepackage{textpos} % textblock
\usepackage{textcomp} % textblock
\usepackage{hyperref}
\usepackage{dirtree}
\hypersetup{colorlinks=true, urlcolor=blue, linkcolor=black} 
% Listing set up
% bash
\lstdefinestyle{bash}{
language=bash,                     % the language of the code
basicstyle=\scriptsize\ttfamily,       % the size of the fonts that are used for the code
numbers=none,%left,                   % where to put the line-numbers
numberstyle=\tiny\color{gray},  % the style that is used for the line-numbers
stepnumber=1,                   % the step between two line-numbers. If it's 1, each line
                          % will be numbered
numbersep=5pt,                  % how far the line-numbers are from the code
backgroundcolor=\color{lightgray},  % choose the background color. You must add \usepackage{color}
showspaces=false,               % show spaces adding particular underscores
showstringspaces=false,         % underline spaces within strings
showtabs=false,                 % show tabs within strings adding particular underscores
frame=lines,%single,                   % adds a frame around the code
rulecolor=\color{black},        % if not set, the frame-color may be changed on line-breaks within not-black text (e.g. commens (green here))
tabsize=2,                      % sets default tabsize to 2 spaces
captionpos=b,                   % sets the caption-position to bottom
breaklines=true,                % sets automatic line breaking
breakatwhitespace=false,        % sets if automatic breaks should only happen at whitespace
title=\lstname,                 % show the filename of files included with \lstinputlisting;
                          % also try caption instead of title
keywordstyle=\color{blue},      % keyword style
commentstyle=\color{dkgreen},   % comment style
stringstyle=\color{mauve},      % string literal style
escapeinside={\%*}{*)},         % if you want to add a comment within your code
morekeywords={}            % if you want to add more keywords to the set
}

% Python
\lstdefinestyle{python}{
language=python,
formfeed=\newpage,
basicstyle=\scriptsize\ttfamily,
commentstyle=\color{deepgreen},%\color{gray},
numbers=left,
numberstyle=\tiny\color{gray},
stepnumber=1,
numbersep=5pt,
extendedchars=true,
inputencoding=utf8x,
backgroundcolor=\color{lightgray},%\color{white},
showspaces=false,
showstringspaces=false,
showtabs=false,
frame=lines,
upquote=true,
tabsize=4,
captionpos=b,
breaklines=true,
breakatwhitespace=false,
title=\lstname,
escapeinside=||,
keywordstyle=\color{deepblue},
emphstyle=\color{deepred},
stringstyle=\color{mauve},
literate={ö}{{\"o}}1
       {ä}{{\"a}}1
       {ü}{{\"u}}1
       {ç}{{\c{c}}}1
%morekeywords={models, lambda, forms}
}

\lstdefinestyle{pythonsmall}{
language=python,
formfeed=\newpage,
basicstyle=\ttfamily\tiny,
commentstyle=\color{deepgreen},%\color{gray},
numbers=left,
numberstyle=\tiny\color{gray},
stepnumber=1,
numbersep=5pt,
extendedchars=true,
inputencoding=utf8x,
backgroundcolor=\color{lightgray},%\color{white},
showspaces=false,
showstringspaces=false,
showtabs=false,
frame=lines,
upquote=true,
tabsize=4,
captionpos=b,
breaklines=true,
breakatwhitespace=false,
title=\lstname,
escapeinside=||,
keywordstyle=\color{deepblue},
emphstyle=\color{deepred},
stringstyle=\color{mauve},
literate={ö}{{\"o}}1
       {ä}{{\"a}}1
       {ü}{{\"u}}1
       {ç}{{\c{c}}}1
%morekeywords={models, lambda, forms}
}

\graphicspath{ {../img/} }
\title[Python for scientific research]{Python for scientific research}
\subtitle{Input, output and the filesystem}
\author{Bram Kuijper}
\institute[]{University of Exeter, Penryn Campus, UK}
\titlegraphic{
\hfill
\includegraphics[width=\textwidth, keepaspectratio]{logo.jpg}}
%=============================================================================%
%=============================================================================%
% Start of Document
%=============================================================================%
%=============================================================================%
\begin{document}

%=============================================================================%
%=============================================================================%
\begin{frame}
\titlepage
\end{frame}

%=============================================================================%
%=============================================================================%
\begin{frame}{What we've done so far}
	\begin{enumerate}\addtolength{\itemsep}{1\baselineskip}
		\item Declare variables using built-in data types and execute operations
		on them
		\item Use flow control commands to dictate the order in which commands are run
		and when
		\item Encapsulate programs into reusable functions, modules and packages
        \item Work with textual data and pattern matching
        \item \textbf{Next} working with files \& the file system 
	\end{enumerate}

\end{frame}

%=============================================================================%
%=============================================================================%

\begin{frame}[fragile]
    \frametitle{Filesystem: introduction}

    \begin{itemize}
        \item Nearly every program will \href{https://docs.python.org/3/tutorial/inputoutput.html\#reading-and-writing-files}{read and write files} (also called file IO; file input and output) \pause 
        \item The file IO workhorse is the \href{https://docs.python.org/3/library/functions.html\#open}{\texttt{open()}} function, which opens a file and returns a file object \pause
        \item Next, there are several utilities to find out information about files and directories
            contained in modules like \href{https://docs.python.org/3/library/os.html?highlight=os\#module-os}{\texttt{os}}, \href{https://docs.python.org/3/library/os.path.html}{\texttt{os.path}} and \href{https://docs.python.org/3/library/shutil.html#module-shutil}{\texttt{shutil}} 
    \end{itemize}
\end{frame}

%=============================================================================%
%=============================================================================%

\begin{frame}[fragile]
    \frametitle{Using \texttt{open()}: writing files}
\begin{itemize}
    \item Open a file for writing, using the \href{https://docs.python.org/3/library/functions.html\#open}{\texttt{open()}} command \pause
\begin{lstlisting}[style=python]
# mode="w" keyword argument: we write to a file
# any existing contents will be overwritten
file_obj = open(file="myfile.txt",mode="w") | \pause |

# write a string to the file
file_obj.write("Hello world!") | \pause |

# always close the file
file_obj.close() | \pause |
\end{lstlisting}
    \item The above listing is bad practice, however: any error in lines 4 - 8 will cause python to quit before \texttt{file\_obj.close()} is called, leaving \texttt{file\_obj} open
\end{itemize}
\end{frame}

%=============================================================================%
%=============================================================================%
\begin{frame}[fragile]
    \frametitle{Using \texttt{open()}: writing files II}
\begin{itemize}
    \item Instead, we need to use a \texttt{with} statement \pause
\begin{lstlisting}[style=python,belowskip=-0.8 \baselineskip]
with open(file="myfile.txt",mode="w") as file_obj:| \pause |
    file_obj.write("Hello world!") | \pause |
\end{lstlisting}
    \item Line 1: the variable \texttt{file\_obj} gets assigned the return value of \texttt{open()} \pause 
    \item Line 2: we perform operations on \texttt{file\_obj} \pause
    \item In case line 2 finishes or ends in an error, a hidden \href{https://docs.python.org/3/reference/datamodel.html#object.__exit__}{\texttt{\_\_exit()\_\_}} function is called on \texttt{file\_obj}, closing it \pause
    \item Bottom line: when using \texttt{with}, \texttt{file\_obj} will not remain open
\end{itemize}
\end{frame}

\begin{frame}[fragile]
    \frametitle{Reading files, using \texttt{open()}}
\begin{itemize}
    \item Open a file for reading, using the \href{https://docs.python.org/3/library/io.html#io.TextIOBase.read}{\texttt{read()}} command \pause
\begin{lstlisting}[style=python]
# mode="r" keyword argument: we read from a file
with open(file="myfile.txt",mode="r") as file_obj:
    # get the file contents as a string
    file_contents = file_obj.read() | \pause |

# process file output
print(file_contents) # Hello World!
\end{lstlisting}
\end{itemize}
\end{frame}

\begin{frame}[fragile]
    \frametitle{Other modes of the \texttt{open()} function}
\begin{itemize}
    \item Different file opening modes can be specified by mode keyword of the \href{https://docs.python.org/3/library/functions.html#open}{\texttt{open()}} function \pause 
\begin{center}
    \rowcolors{1}{}{lightgray} %-- this indicates the change in odd and pair rows
    \begin{tabular}{|l|l|}\hline
        Flag & File operation \\ \hline
        \texttt{"w"}
    & Write to a file, file will be truncated first \\
        \texttt{"r"} & Reading from a file \\
        \texttt{"r+"} & Reading and writing to a file (no truncation) \\
        \texttt{"a"} & Append to a file \\
        \texttt{"a+"} & Read from and write (by appending) to a file \\
        \texttt{"x"} & Exclusive creation, fails if file exists \\ \hline
    \end{tabular}
\end{center}
\end{itemize}
\end{frame}

\begin{frame}[fragile]
    \frametitle{Appending to a file}
\begin{itemize}
    \item Append text to a previously opened file using \href{https://docs.python.org/3/library/functions.html#open}{\texttt{mode="a+"}}: 
\begin{lstlisting}[style=python]
# first write something to a new file
with open(file="myfile.txt",mode="w") as file_obj: | \pause |
    file_obj.write("I wrote this to a file!")

with open(file="myfile",mode="a+") as file_obj: | \pause |
    file_obj.write("\nAnd now I also wrote this!")

    # get the file contents as a string
    file_contents = file_obj.read() | \pause |

# process file output
print(file_contents) # Nothing!
\end{lstlisting} \pause 
    \item No output because the internal file pointer used by \href{https://docs.python.org/3/library/io.html#io.TextIOBase.read}{\texttt{file.read()}} is at the end of the file!
\end{itemize}
\end{frame}


\begin{frame}[fragile]
    \frametitle{Position of file pointer}
            \begin{figure}\centering%
                \vspace{1ex}%
                \includegraphics[width = 75mm]{file_pointer1.pdf}
            \end{figure}
\end{frame}

\begin{frame}[fragile]
    \frametitle{Position of file pointer}
            \begin{figure}\centering%
                \vspace{1ex}%
                \includegraphics[width = 75mm]{file_pointer2.pdf}
            \end{figure}
\end{frame}


\begin{frame}[fragile]
    \frametitle{Appending to a file}
\begin{itemize}
    \item Solution: use the \href{https://docs.python.org/3/library/io.html#io.TextIOBase.seek}{\texttt{file.seek()}} command to change the internal file pointer position \pause 
\begin{lstlisting}[style=python]
# first write something to a new file:
with open(file="myfile.txt",mode="w") as file_obj: | \pause |
    file_obj.write("I wrote this to a file!")

# open file again, now to append and read:
with open(file="myfile.txt",mode="a+") as file_obj: | \pause |
    file_obj.write("\nAnd now I also wrote this!")

    # move the file pointer to the 0th byte (start)
    # of the file:
    file_obj.seek(0)

    # get the file contents as a string
    file_contents = file_obj.read() | \pause |

# process file output
print(file_contents) # output: I wrote this...
\end{lstlisting} \pause 
\end{itemize}
\end{frame}

%=============================================================================%
%=============================================================================%

\begin{frame}[fragile]
    \frametitle{Working with the filesystem: key modules}
\begin{itemize}\addtolength{\itemsep}{\baselineskip}
\item Reading and writing files
    \begin{itemize}
        \item The \href{https://docs.python.org/3/library/functions.html#open}{\texttt{open()}} command \pause
    \end{itemize}
    \item Work with \emph{names} of directories and files (pathnames) \pause
    \begin{itemize}
        \item Newer, object-oriented \href{https://docs.python.org/3/library/pathlib.html#module-pathlib}{\texttt{Pathlib}} module (Python 3.5+) \pause
        \item This course: conventional \href{https://docs.python.org/3.7/library/os.path.html}{\texttt{os.path}} module \pause
    \end{itemize}
\item Access the filesystem (disks, file descriptors, directories, etc): \pause
    \begin{itemize}
        \item Basic filesystem tasks: the \href{https://docs.python.org/3.7/library/os.html#module-os}{\texttt{os}} module (user access rights, removing directories, etc) \pause
        \item High-level operations: the \href{https://docs.python.org/3/library/shutil.html}{\texttt{shutil}} module (copying files, removing directory trees, etc)  \pause
    \end{itemize}
\end{itemize}
\end{frame}

%=============================================================================%
%=============================================================================%

\begin{frame}[fragile]
    \frametitle{Working with filesystems: key functions}
\begin{itemize}\addtolength{\itemsep}{\baselineskip}
        \item Some important functions of the \href{https://docs.python.org/3.7/library/os.html#module-os}{\texttt{os}} module:\pause
            \begin{itemize}\addtolength{\itemsep}{\baselineskip}
                \item \href{https://docs.python.org/3.7/library/os.html#os.getcwd}{\texttt{os.getcwd()}} obtains current working directory \pause
                \item \href{https://docs.python.org/3.7/library/os.html#os.chdir}{\texttt{os.chdir(dir)}} changes current working directory to \texttt{dir} \pause
                \item \href{https://docs.python.org/3.7/library/os.html#os.rmdir}{\texttt{os.rmdir(dir)}} removes directory \texttt{dir} \pause
                \item \href{https://docs.python.org/3.7/library/os.html#os.mkdir}{\texttt{os.mkdir(dir)}} makes directory \texttt{dir} \pause
                \item \href{https://docs.python.org/3.7/library/os.html#os.listdir}{\texttt{os.listdir(dir=".")}} list files in directory \texttt{dir} \pause
                \item \href{https://docs.python.org/3.7/library/os.html#os.walk}{\texttt{os.walk(dir)}} loop through all subdirectories and files in directory \texttt{dir} \pause
        \end{itemize}
    \end{itemize}
\end{frame}

%=============================================================================%
%=============================================================================%

\begin{frame}[fragile]
    \frametitle{Working with filesystems: key functions II}
        \begin{itemize}\addtolength{\itemsep}{\baselineskip}
            \item Some important functions of the \href{https://docs.python.org/3.7/library/os.html#module-os.path}{\texttt{os.path}} module:\pause
                \begin{itemize}\addtolength{\itemsep}{\baselineskip}
                        \item \href{https://docs.python.org/3.7/library/os.path.html#os.path.exists}{\texttt{os.path.exists(path)}} checks whether path exists \pause
                        \item \href{https://docs.python.org/3.7/library/os.path.html#os.path.expanduser}{\texttt{os.path.expanduser("\textasciitilde")}} provides home directory name of current user \pause
                        \item \href{https://docs.python.org/3.7/library/os.path.html#os.path.join}{\texttt{os.path.join(path1,path2)}} concatenates path1 and path2  \pause
                        \item \href{https://docs.python.org/3.7/library/os.path.html#os.path.split}{\texttt{os.path.split(path)}} splits path \texttt{a/b/c.txt} into \texttt{(a/b, c.txt)} \pause
                        \item \href{https://docs.python.org/3.7/library/os.path.html#os.path.splitext}{\texttt{os.path.splitext(path)}} splits path \texttt{a/b/c.ext} into \texttt{(a/b/c, ext)} \pause
                \end{itemize}
        \end{itemize}
\end{frame}

%=============================================================================%
%=============================================================================%


\begin{frame}[fragile]
    \frametitle{Working with paths 1}
    \begin{itemize}
        \item Task: what is the current working directory? 
            \begin{itemize}
                \item Access the filesystem using the \href{https://docs.python.org/3.7/library/os.html#module-os}{\texttt{os}} module: \pause
            \end{itemize}
\begin{lstlisting}[style=python]
import os # import modules | \pause |

wdir = os.getcwd() # returns pathname (as string) | \pause |
print(wdir) # e.g., "C:\\" 
\end{lstlisting}
    \end{itemize}
\end{frame}
%=============================================================================%
%=============================================================================%


\begin{frame}[fragile]
    \frametitle{Working with paths 2}
    \begin{itemize}
        \item Task: change the current working directory to the home directory: \pause
    \begin{itemize}
            \item Access the filesystem using the \href{https://docs.python.org/3.7/library/os.html#module-os}{\texttt{os}} module 
            \item Manipulate path names using the \href{https://docs.python.org/3.7/library/os.path.html}{\texttt{os.path}} module \pause
    \end{itemize}
\begin{lstlisting}[style=python]
import os, os.path # import modules | \pause |

# Get homedir through expanduser()
# "~" denotes homedir in Unix (but this works on Windows too)
home_dir = os.path.expanduser("~") | \pause |
print(home_dir) # "C:\\Users\\foo323" | \pause |

# change the directory to the homedir
os.chdir(home_dir) | \pause |

# check the result
print(os.getcwd()) # "C:\\Users\\foo323"
\end{lstlisting}
    \end{itemize}
\end{frame}

%=============================================================================%
%=============================================================================%
\begin{frame}[fragile]
    \frametitle{Finding files}
    \begin{itemize}
        \item Task: in your home directory, find all subdirectories (non-nested) which contain whitespace
\begin{lstlisting}[style=python]
import re, os, os.path # import the modules | \pause |

# go to home dir
home_dir = os.path.expanduser(path="~") | \pause |
os.chdir(path=home_dir) | \pause |

# list all files in home dir
flist = os.listdir() #["Desktop", "Downloads",...] | \pause |

# declare list containing dirs with spaces
dirs_with_spaces = [] | \pause |

for file_i in flist: | \pause |
    # check for white space in filename and whether file is directory
    if re.search(pattern=r"\s",string=file_i) != None and os.path.isdir(file_i): | \pause |
        dirs_with_spaces.append(file_i)
\end{lstlisting} \pause
    \end{itemize}
\end{frame}

%=============================================================================%
%=============================================================================%

\begin{frame}[fragile]
    \frametitle{Finding files 2}
    \begin{itemize}\addtolength{\itemsep}{1\baselineskip}
        \item Task: list all files \emph{anywhere} within the home directory which have a size larger than 50 kB \pause
        \item Iterate over all files in any subdirectory within the home directory, using \href{https://docs.python.org/3/library/os.html#os.walk}{\texttt{os.walk()}}: \pause
    \begin{itemize}
        \item \href{https://docs.python.org/3/library/os.html#os.walk}{\texttt{os.walk()}} walks the directory tree, returning a tuple for each directory with \texttt{(parent\_dir, subdirectories, files)}
    \end{itemize}
    \end{itemize}
\end{frame}

%=============================================================================%
%=============================================================================%

\begin{frame}[fragile]
    \frametitle{Finding files using \texttt{os.walk()}}
    \begin{itemize}
        \item Task: list all files \emph{anywhere} within the home directory which have a size larger than 1 Mb \pause
        \item
    Say, we have the following directory tree
            \dirtree{%
    .1 C:.
    .2 Users.
    .2 Public.
    .3 \textbf{foo323}.
    .4 Desktop.
    .4 Downloads.
    .5 citation.txt (5 Kb).
    .5 boring\_paper.pdf (5 Mb).
    .4 Scripts.
    .4 groceries.docx (10 Kb).
    .4 manuscript.docx (2 Mb).
            } \pause
        \item We should return \texttt{Downloads/boring\_paper.pdf} and \texttt{manuscript.docx}
        \end{itemize}
\end{frame}

%=============================================================================%
%=============================================================================%

\begin{frame}[fragile]
    \frametitle{Finding files using \texttt{os.walk()}}
    \begin{itemize}
        \item To illustrate how \href{https://docs.python.org/3.7/library/os.html#os.walk}{os.walk()} works, we iterate over the home directory and then exit the loop through \texttt{break}:
\begin{lstlisting}[style=python]
import os 

# get home directory
homedir = os.path.expanduser(path="~") | \pause |

# iterate over all files nested in the home directory
for parent_dir, subdirs, files in os.walk(homedir): | \pause |
    print(parent_dir) # C:\Users\Public\foo323 | \pause |
    print(subdirs ) # ['Desktop', 'Downloads', 'Scripts'] | \pause |
    print(files) # ['groceries.docx', 'manuscript.docx'] | \pause |

    # quit after the first iteration
    break

\end{lstlisting}
    \end{itemize}
\end{frame}

\begin{frame}[fragile]
    \frametitle{Finding files using \texttt{os.walk()}}
    \begin{itemize}
        \item To illustrate how \href{https://docs.python.org/3.7/library/os.html#os.walk}{os.walk()} works, we iterate over the home directory and then exit the loop through \texttt{break}:
\begin{lstlisting}[style=pythonsmall]
import os, os.path

# get home directory
homedir = os.path.expanduser("~") | \pause |

# make a list to store the files
files_larger_1mb = []

# iterate over all files nested in the home directory
for parent_dir, subdirs, files in os.walk(homedir): | \pause |

    for file in files: | \pause |

        # get the full path name
        full_path = os.path.join(parent_dir, file) | \pause |

        if os.path.exists(full_path): | \pause |
            size = os.path.getsize(full_path) | \pause |

            if size / 1024 > 1000:
                files_larger_1mb += [full_path] | \pause |

# print folder contents
print(files_larger_1mb)
\end{lstlisting}
    \end{itemize}
\end{frame}

\begin{frame}[fragile]
    \frametitle{Copying files using \texttt{shutil}}
\begin{itemize}
    \item Task: make a file and then copy it to another file using \href{https://docs.python.org/3/library/shutil.html}{\texttt{shutil}}
\begin{lstlisting}[style=python]
import shutil 

# store a filename 
filename = "new_file.txt" | \pause |

with open(file="new_file.txt",mode="w") as f:| \pause |
    f.write("some text") | \pause |

# now copy using shutil
shutil.copy(filename, "another_new_file.txt") | \pause |

# list all the files in the current directory
# to see whether copied file exists
print(os.listdir(".")) | \pause |

\end{lstlisting}
\end{itemize}
\end{frame}


% End of Document
%=============================================================================%
%=============================================================================%
\end{document}
