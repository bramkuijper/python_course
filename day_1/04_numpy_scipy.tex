%=============================================================================%
% Author: 	John Joseph Valletta, Bram Kuijper
% Date: 	14/03/2017, 27/03/2019
% Title: 	Python workshop: numpy scipy 
%=============================================================================%

%=============================================================================%
% Preamble
%=============================================================================%
% Libraries

\documentclass[xcolor=table]{beamer}
\usepackage{beamerthemeshadow}
\usepackage{mathptmx}
\usepackage{helvet}
\usepackage[]{graphicx}
\usepackage{array}
\usepackage{color}
\definecolor{dkgreen}{rgb}{0,0.6,0}
\definecolor{gray}{rgb}{0.5,0.5,0.5}
\definecolor{mauve}{rgb}{0.58,0,0.82}
\definecolor{deepblue}{rgb}{0,0,0.5}
\definecolor{deepred}{rgb}{0.6,0,0}
\definecolor{deepgreen}{rgb}{0,0.5,0}
\definecolor{lightgray}{rgb}{0.92,0.92,0.92}
\usepackage{listings} % to insert code
\usepackage{textpos} % textblock
\usepackage{dirtree} % textblock
\usepackage{hyperref}
\hypersetup{colorlinks=true, urlcolor=blue, linkcolor=black} 
% Listing set up
% bash
\lstdefinestyle{bash}{
language=bash,                     % the language of the code
basicstyle=\scriptsize\ttfamily,       % the size of the fonts that are used for the code
numbers=none,%left,                   % where to put the line-numbers
numberstyle=\tiny\color{gray},  % the style that is used for the line-numbers
stepnumber=1,                   % the step between two line-numbers. If it's 1, each line
                          % will be numbered
numbersep=5pt,                  % how far the line-numbers are from the code
backgroundcolor=\color{lightgray},  % choose the background color. You must add \usepackage{color}
showspaces=false,               % show spaces adding particular underscores
showstringspaces=false,         % underline spaces within strings
showtabs=false,                 % show tabs within strings adding particular underscores
frame=lines,%single,                   % adds a frame around the code
rulecolor=\color{black},        % if not set, the frame-color may be changed on line-breaks within not-black text (e.g. commens (green here))
tabsize=2,                      % sets default tabsize to 2 spaces
captionpos=b,                   % sets the caption-position to bottom
breaklines=true,                % sets automatic line breaking
breakatwhitespace=false,        % sets if automatic breaks should only happen at whitespace
title=\lstname,                 % show the filename of files included with \lstinputlisting;
                          % also try caption instead of title
keywordstyle=\color{blue},      % keyword style
commentstyle=\color{dkgreen},   % comment style
stringstyle=\color{mauve},      % string literal style
escapeinside={\%*}{*)},         % if you want to add a comment within your code
morekeywords={}            % if you want to add more keywords to the set
}

% Python
\lstdefinestyle{python}{
language=python,
formfeed=\newpage,
basicstyle=\scriptsize\ttfamily,
commentstyle=\color{deepgreen},%\color{gray},
numbers=left,
numberstyle=\tiny\color{gray},
stepnumber=1,
numbersep=5pt,
backgroundcolor=\color{lightgray},%\color{white},
showspaces=false,
showstringspaces=false,
showtabs=false,
frame=lines,
tabsize=4,
captionpos=b,
breaklines=true,
breakatwhitespace=false,
title=\lstname,
escapeinside=||,
keywordstyle=\color{deepblue},
emphstyle=\color{deepred},
stringstyle=\color{deepgreen}
%morekeywords={models, lambda, forms}
}

\graphicspath{ {../img/} }
\title[Python for scientific research]{Python for scientific research}
\subtitle{Number crunching with numpy and scipy}
\author{Bram Kuijper}
\institute[]{University of Exeter, Penryn Campus, UK}
\titlegraphic{
\hfill
\includegraphics[width=\textwidth, keepaspectratio]{logo.jpg}}
%=============================================================================%
%=============================================================================%
% Start of Document
%=============================================================================%
%=============================================================================%
\begin{document}

%=============================================================================%
%=============================================================================%
\begin{frame}
\titlepage
\end{frame}

%=============================================================================%
%=============================================================================%
\begin{frame}{What we've done so far}

	\begin{enumerate}\addtolength{\itemsep}{1\baselineskip}
		\item Declare variables using built-in data types and execute operations
		on them
		\item Use flow control commands to dictate the order in which commands are run
		and when
		\item Encapsulate programs into reusable functions, modules and packages
		\item \textbf{Next}: Number crunching using \texttt{NumPy/SciPy}
	\end{enumerate}

\end{frame}

%=============================================================================%
%=============================================================================%
\begin{frame}{Motivation}

\begin{itemize}\addtolength{\itemsep}{\baselineskip}
	\item<1-> We are typically faced by numerical tasks, 
	such as, computing Pearson's correlation coefficient
	or generating random numbers

	\item<2-> The \texttt{NumPy} (numeric python) package provides
	a comprehensive set of mathematical data structures (e.g vector, matrix)
	and functions (e.g trigonometry, random number generator) 

	\item<3-> The \texttt{SciPy} (scientific python) library builds on top
	of \texttt{NumPy} to provide a collection of numerical algorithms for: 
	\begin{itemize}
		\item<4-> Statistics (e.g correlation coefficient)
		\item<5-> Signal processing (e.g Fourier transform, filtering)
		\item<6-> Solving differential equations
		\item<7-> Optimisation
		\item<8-> $\ldots$
	\end{itemize}
\end{itemize}
\end{frame}

%=============================================================================%
%=============================================================================%
\begin{frame}[fragile]
\frametitle{A taste of NumPy}

\begin{lstlisting}[style=python]
import numpy as np

# 1D array/vector
x = np.array([1, 2, 3, 4])
x.min() # return min of array 
x.max() # return max of array
x.sum() # sum all elements in array

# 2D array
x = np.array([[1, 2, 3], [4, 5, 6], [7, 8, 9]])
x*x # element-by-element multiplication
x.shape # return dimensions of matrix
np.dot(x, x) # matrix multiplication
np.mat(x)*np.mat(x) # matrix multiplication (cast as matrix)
\end{lstlisting}

\end{frame}

%=============================================================================%
%=============================================================================%
\begin{frame}[fragile]
\frametitle{A taste of NumPy}

\begin{lstlisting}[style=python]
# Generate random numbers
np.random.rand() # from a uniform distribution
np.random.randn() # from a normal distribution
np.random.randn(5, 5) # 5 x 5 matrix of random numbers

# Create number sequences
np.arange(0, 1, 0.1) # 0 to 1 in steps of 0.1
np.linspace(0, 1, 100) # 100 values between 0 and 1
np.logspace(0, 1, 10) # 10 values between 10^0 and 10^1
\end{lstlisting}

\end{frame}

%=============================================================================%
%=============================================================================%
\begin{frame}[fragile]
\frametitle{A taste of SciPy}

\begin{lstlisting}[style=python]
import scipy.stats as sp

# Create two random arrays
x1 = np.random.randn(30)
x2 = np.random.randn(30)

# Correlation coefficientss
sp.pearsonr(x1, x2) # pearson correlation
sp.spearmanr(x1, x2) # spearman correlation
sp.kendalltau(x1, x2) # kendall correlation

# Statistical tests
sp.ttest_ind(x1, x2) # independent t-test
sp.mannwhitneyu(x1, x2) #  Mann-Whitney rank test 
sp.wilcoxon(x1, x2) # Wilcoxon signed-rank test

# Least-squares regression
sp.linregress(x1, x2)
\end{lstlisting}

\end{frame}

%=============================================================================%
%=============================================================================%
\begin{frame}[fragile]
\frametitle{Predator prey equations (Lotka Volterra)}

%Prey population = growth rate - rate at which it is preyed
%Predator population = - natural death + food supply

\begin{align*}
    \frac{\mathrm{d}u}{\mathrm{d}t} &= \alpha u - \beta uv\\[1em]
\frac{\mathrm{d}v}{\mathrm{d}t} &= -\gamma v + \delta uv 
\end{align*}

% \begin{align*}
% \text{Prey population:}& \frac{\d{u}}{\d{t}} &=& \underbrace{\alpha u}_\text{own growth rate} - \underbrace{\beta uv}_\text{rate at which it's preyed upon}\\
% \text{Prey population:}& \frac{\d{v}}{\d{t}} &=& \underbrace{-\gamma v}_\text{natural death of predators} + \underbrace{\delta uv}_\text{growth of predator population} 
% \end{align*}




Where:

\begin{itemize}
\item $u$: is the number of prey (e.g rabbits)
\item $v$: is the number of predators (e.g foxes)
%$\frac{\d{x}}{\d{t}}$: prey population growth rate\\
%$\frac{\d{y}}{\d{t}}$: predator population growth rate\\
\item $\alpha$: prey growth rate in the absence of predators
\item $\beta$: dying rate of prey due to predation
\item $\gamma$: dying rate of predators in the absence of prey
\item $\delta$: predator growth rate when consuming prey
\end{itemize}

% Prey population = growth rate - rate at which it is preyed
% Predatory population = - natural death + food supply

\end{frame}

%=============================================================================%
%=============================================================================%
\begin{frame}[fragile]
\frametitle{Predator prey equations in Python}

\vspace{-0.4cm}
\scriptsize
\begin{align*}
    \frac{\mathrm{d}u}{\mathrm{d}t} &= \alpha u - \beta uv\\[0.5em]
\frac{\mathrm{d}v}{\mathrm{d}t} &= -\gamma v + \delta uv 
\end{align*}
\normalsize


\begin{lstlisting}[style=python]
def predator_prey(x, t):
    """
	Predator prey model (Lotka Volterra)
    """
    # Constants
    alpha = 1
    beta = 0.1
    gamma = 1.5
    delta = 0.075
    
    # x = [u, v] describes prey and predator populations
    u, v = x
    
    # Define differential equation (u = x[0], v = x[1])
    du = alpha*u - beta*u*v
    dv = -gamma*v + delta*u*v
    
    return du, dv
\end{lstlisting}

\end{frame}

%=============================================================================%
%=============================================================================%
\begin{frame}[fragile]
\frametitle{Solve differential equations}

\begin{lstlisting}[style=python] 
from scipy.integrate import odeint

time = np.linspace(0, 35, 1000) # time vector
init = [10, 5] # initial condition: 10 prey, 5 predators
x = odeint(predator_prey, init, time) # solve
\end{lstlisting}

\vspace{-.8cm}

\visible<2->{
\begin{center}
	\includegraphics[width=.65\textwidth]{predator_prey.pdf}
\end{center}}

\end{frame}

%=============================================================================%
%=============================================================================%
\begin{frame}[fragile]
\frametitle{Fourier transform}

\begin{lstlisting}[style=python]
from scipy.fftpack import fftfreq, fft

# Create frequency vector
N = len(time)
freq = fftfreq(N, np.mean(np.diff(time)))
freq = freq[range(int(N/2))]

# Compute Fast Fourier Transform
y = fft(x[:, 0])/N # compute and normalise fft
y = y[range(int(N/2))] # keep only positive frequencies
\end{lstlisting}

\end{frame}

%=============================================================================%
%=============================================================================%
\begin{frame}[fragile]
\frametitle{Fourier transform}

\begin{center}
	\includegraphics[width=.8\textwidth]{fourier.pdf}
\end{center}

\end{frame}

%=============================================================================%
%=============================================================================%
% End of Document
%=============================================================================%
%=============================================================================%
\end{document}
